% Offizielle Beispieldatei für beamer-Vorlage aus tubslatex Version 0.3beta2
\documentclass[fleqn,11pt,aspectratio=43]{beamer}

\usepackage[ngerman]{babel}
\usepackage[utf8x]{inputenc}
\usepackage{graphicx}
\usepackage{minted}
\usetheme[%
  %nexus,%        Nexus Fonts benutzen
  %lnum,%         Versalziffern verwenden
  %cmyk,%<rgbprint>,          Auswahl des Farbmodells
  blue,%<orange/green/violet> Auswahl des Sekundärfarbklangs
  dark,%<light,medium>        Auswahl der Helligkeit
  %colorhead,%    Farbig hinterlegte Kopfleiste
  %colorfoot,%    Farbig hinterlegt Fußleiste auf Titelseite
  colorblocks,%   Blöcke Farbig hinterlegen
  %nopagenum,%    Keine Seitennumer in Fußzeile
  %nodate,%       Kein Datum in Fußleiste
  tocinheader,%   Inhaltsverzeichnis in Kopfleiste
  %tinytocinheader,% kleines Kopfleisten-Inhaltsverzeichnis
  %widetoc,%      breites Kopfleisten-Inhaltsverzeichnis
  %narrowtoc,%    schmales Kopfleisten-Inhaltsverzeichnis
  %nosubsectionsinheader,%  Keine subsections im Kopfleisten-Inhaltsverzeichnis
  %nologoinfoot,% Kein Logo im Fußbereich darstellen
  ]{tubs}

% Titelseite
\title{Dynamic Parallelism in CUDA}
\subtitle{The Mandelbrot Example}
\author{Marc Kassubeck}
% Titelgrafik, automatisch beschnitten, Weitere Optionen: <scaled/cropx/cropy>
% \titlegraphic[cropped]{\includegraphics{infozentrum.jpg}}
\titlegraphic[scaled]{\includegraphics{titlepicture.jpg}}

% Logo, dass auf Titelseiten oben rechts und auf Inthaltsseiten unten rechts
% dargestellt wird. Es wird jeweils automatisch skliert
\logo{\includegraphics{wire.jpeg}}
%\logo{Institut für Unkreativität\\und Schreibschwäche}

\begin{document}
\setminted{fontsize=\scriptsize, autogobble, breaklines, frame=single, tabsize=4}

\begin{frame}[plain]
\titlepage
\end{frame}

\begin{frame}[plain]
  \tableofcontents
\end{frame}


\section{The Mandelbrot Set}

\begin{frame}{Definition of the Mandelbrot Set}
	\begin{itemize}
		\item is a well known fractal
		\item consists of two parts
		\item definition of a series
	\end{itemize}	
	\begin{align*}
		z_0 &= c\\
		z_{n+1} &= z_n^2 + c
	\end{align*}
	\begin{itemize}
		\item defininition of the set in the complex plane
	\end{itemize}
	\begin{align*}
		M &= \{c \in \mathbb{C}: \exists R\ \forall n: \left| z_n \right| < R \}
	\end{align*}
	\begin{itemize}
		\item implementation won't check all $n$, but all $n$ up to a fixed number
	\end{itemize}
\end{frame}

\subsection{The Escape Time Algorithm}

\begin{frame}{The Escape Time Algorithm}
	\begin{itemize}
		\item one of the easiest algorithms for computing the Mandelbrot Set
		\item computes the \textit{dwell} for each pixel (point in the complex plane) according to the series definition
		\item for a fixed maximum number of iterations the series is computed and it is checked, if the value of \textit{dwell} 'escapes' outside a circle of radius 2
		\item this is correct, as the series will diverge, if one element lies outside this circle
		\item if the point 'escapes', the color value of the point will be the \textit{dwell}
		\item otherwise it will be black
	\end{itemize}
\end{frame}

\section{Computing the Mandelbrot Set in CUDA}

\begin{frame}[fragile]{The pixel dwell}
	This code block computes the dwell for a single pixel position:
	\begin{minted}{c++}
		#define MAX_DWELL 512
		// w, h --- width and hight of the image, in pixels
		// cmin, cmax --- coordinates of bottom-left and top-right image corners
		// x, y --- coordinates of the pixel
		__host__ __device__ int pixel_dwell(int w, int h, complex cmin, complex cmax, int x, int y) {
			complex dc = cmax - cmin;
			float fx = (float)x / w, fy = (float)y / h;
			complex c = cmin + complex(fx * dc.re, fy * dc.im);
			complex z = c; 
			int dwell = 0;
		
			while(dwell < MAX_DWELL && abs2(z) < 2 * 2) { 
				z = z * z + c;
				dwell++;
			}
			return dwell;
		}
	\end{minted}
\end{frame}

\begin{frame}[fragile]{Invoking the pixel-dwell-kernel}
	The easiest way to generate the Mandelbrot Set is to invoke the \textit{pixel\_dwell}-kernel for each pixel
	\begin{minted}{c++}
		__global__ void mandelbrot_k(int *dwells, int w, int h, complex cmin, complex cmax) {
			int x = threadIdx.x + blockDim.x * blockIdx.x;
			int y = threadIdx.y + blockDim.y * blockIdx.y;
			if(x < w && y < h)
				dwells[y * w + x] = pixel_dwell(w, h, cmin, cmax, x, y);
		}
	\end{minted}
	The kernel launch on the CPU
	\begin{minted}{c++}
	  int w = 4096, h = 4096;
	  dim3 bs(64, 4), grid(divup(w, bs.x), divup(h, bs.y));
	  mandelbrot_k <<<grid, bs>>>(d_dwells, w, h, complex(-1.5, 1), complex(0.5, 1));
	\end{minted}
\end{frame}

\section{Extending the Algorithm with Dynamic Parallelism}

\begin{frame}{Extending the Algorithm with Dynamic Parallelism}
	\begin{itemize}
		\item Extension relies on a mathematical fact
		\item The set is \textit{connected}
		\item Between any two parts of the set, there is a path, that lies wholly inside the set
		\item Thus, a region (of any shape) lies within the Mandelbrot Set, iff its border lies within the set
		\item More genrally, if the border of the region has a constant dwell, then every pixel in the region has the same dwell
		\item This saves computation time
	\end{itemize}
\end{frame}

\subsection{Compiling the Algorithm for Dynamic Paralellism}

\end{document}
